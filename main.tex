\documentclass[justified]{tufte-handout}
\usepackage[fontsize=8pt]{fontsize}

\usepackage{comment}

\begin{comment}
\makeatletter % Paragraph indentation and separation for normal text
\renewcommand{\@tufte@reset@par}{%
	\setlength{\RaggedRightParindent}{0pc}%
	\setlength{\JustifyingParindent}{0pc}%
	\setlength{\parindent}{0pc} \setlength{%
		\parskip}{0pt} } \@tufte@reset@par%

% Paragraph indentation and separation for marginal text
\renewcommand{\@tufte@margin@par}{%
	\setlength{\RaggedRightParindent}{0pc}%
	\setlength{\JustifyingParindent}{0pc}%
	\setlength{\parindent}{0pc} \setlength{%
		\parskip}{0pt} } \makeatother%
\end{comment}

\renewcommand\allcapsspacing[1]{{\addfontfeature{LetterSpace=15}#1}}
\renewcommand\smallcapsspacing[1]{{\addfontfeature{LetterSpace=10}#1}}

\newcommand\scalemath[2]{\scalebox{#1}{\mbox{\ensuremath{\displaystyle #2}}}}

\usepackage[french]{babel}

\usepackage{amssymb}
\usepackage{amsmath}
\usepackage{amsthm}

\usepackage{mathpartir}

\usepackage{luamplib}
\everymplib{ beginfig(0); }
\everyendmplib{ endfig; }

\usepackage{unicode-math}
\usepackage{libertinus-otf}
\usepackage[Libertinus]{mycd}
%\setmainfont{LinLibertine_Rah.ttf}[Ligatures={Common,Rare,TeX}, Numbers=OldStyle]
%\setmathfont{LibertinusMath-Regular.otf}
%\setmonofont{JuliaMono-Bold.ttf}

%\usepackage{tcolorbox}
%\newtcolorbox{defbox}{
%    colback=yellow!4!white,
%    boxrule=0.5pt,
%    sharp corners,
%}

\usepackage{tikz}
\usepackage{tikz-cd}
\tikzcdset{ arrow style=math font, diagrams={>=stealth}}

\newtheorem{definition}{Définition}
\newtheorem{prop}{Proposition}

\def\msf{\mathsf}
\def\mbf{\mathbold}
\def\mbbf{\mathbb}
\def\mcf{\mathcal}

\title{\huge Théorie des catégories, Notes}
\author{mechap}

\begin{document}
\maketitle

\begin{abstract}
	La théorie des catégories étudie les structures mathématiques et leurs relations.
	En ce sens, nous pouvons la voir comme un ensemble d'outils permettant de décrire des correspondances et relations que des structures mathématiques abstraites entretiennent.
	Il s'agit donc à l'instar de la théorie des ensembles non pas de se focaliser sur des éléments appelés objets mais bien sur les relations entre ces objets : les homomorphismes entre eux.

	Ce document constitue un recueil de mes notes et pensées sur ce sujet. En ce
	sens, les références ainsi que les analogies qui y seront établies seront
	subjectives.
\end{abstract}

\hrulefill

\section{Introduction}

Une catégorie n'est pas moins qu'une généralisation d'un multigraphe, i.e., une
classe d'objets et de morphismes entre eux.

\marginnote{Une catégorie est dite « petite » si $\msf{Ob}(C)$ et $\msf{Mor}(C)$ sont des ensembles.
	Dans le cas contraire, $C$ est une catégorie « large ».
	Similairement, une catégorie $C$ est localement petite si pour tout objets $X, Y \in C$, la collection $\msf{Hom}_C(X, Y)$ des morphismes $X \rightarrow Y$ est un ensemble.
}
\begin{definition}[Catégories]
	Une catégorie $C$ est une classe d'objets $\msf{Ob}(C)$ et de morphismes $\msf{Mor}(C)$ satisfaisant :
	\begin{itemize}
		\item[(i)] Pour chaque morphisme $f$, il y a des objets $\msf{dom}(f) = A$ et $\msf{cod}(f) = B$ appelés respectivement domaine et codomaine de $f$.
			Dans ce cas, nous écrivons $f : A \rightarrow B$.
		\item[(ii)] Soit deux morphismes $f : A \rightarrow B$ et $g : B \rightarrow C$, il existe un morphisme $g \circ f : A \rightarrow C$, i.e., la composition de $f$ et $g$.
		\item[(iii)] Soit un objet $A$, il existe un morphisme identité $1_A : A \rightarrow A$ de telle sorte que pour tout morphisme $f : A \rightarrow$, $f \circ 1_a = f = 1_B \circ f$.
		\item[(iv)] La composition de morphisme est associative : soit trois morphismes $f : A \rightarrow B$, $g : B \rightarrow C$, et $h : C \rightarrow D$ \[ (f \circ g) \circ h = f \circ (g \circ h) \]
	\end{itemize}
\end{definition}

\marginnote{
	\begin{itemize}
		\item[$\bullet$] Les isomorphismes dans la catégorie $\msf{Set}$ des ensembles sont précisément des bijections.
		\item[$\bullet$] Les isomorphismes dans $\msf{Group}$, $\msf{Ring}$, $\msf{Field}$ sont des homomorphismes bijectifs.
		\item[$\bullet$] Les isomorphismes dans la catégories $\msf{Top}$ des espaces topologiques sont des homéomorphismes, i.e., des applications continues avec des inverses continus.
			C'est une propriété plus restrictive qu'une simple bijection continue.
		\item[$\bullet$] Les isomorphismes dans la catégorie $\msf{Htpy}$ sont des équivalences homotopiques.
	\end{itemize}
}
\begin{definition}[Isomorphismes d'objets]
	Soit $C$ une catégorie, un morphisme $f : A \rightarrow B$ est un isomorphisme si et seulement si il existe un morphisme $g : B \rightarrow A$ satisfaisant $f \circ g = 1_B$ et $g \circ f = 1_A$.
	Dans ce cas, $f$ et $g$ sont des inverses (on considère que $f^{-1} = g$) et $A$ est isomorphe à $B$ ($A \simeq B$).
\end{definition}

Un exemple de catégorie essentiel est l'ensemble partiellement ordonné :
$\msf{Poset}$. Par conséquent, les objets de cette catégorie sont simplement
les élément de l'ensemble, et les morphismes $f : x \rightarrow y$
correspondent à la relation d'équivalence $x \leq y$.

\vspace{5pt}

Ce sera également utile de définir les catégories suivantes : $\msf{0}$ est la
catégorie vide (ne possédant aucun objets et morphismes\sidenote{Par analogie
	avec la théorie des types dépendants, où $\msf{0}$ est le type ne possédant
	aucun élément.

	Par l'isomorphisme de Curry Howard, \centering\inferrule{\Gamma \vdash P : A
		\rightarrow \mathcal{U}}{\Gamma \vdash f : \prod_{a : A} P(a) \rightarrow 0} \\
	ne peut être inféré. }), $\msf{1}$ est la catégorie avec un seul objet et le
morphisme identité et $\msf{2}$, la catégorie représentant par le diagramme
suivant.

\begin{center}
	\begin{tikzcd}
		A \arrow[r, "f"]\arrow[loop left, "\msf{id}_A"] & B \arrow[loop right, "\msf{id}_B"]
	\end{tikzcd}
\end{center}

Enfin, un groupe individuel est lui-même une catégorie avec exactement un seul
objet, où tous les morphismes sont des isomorphismes. Soit un groupe $G$, cette
catégorie est notée $\mbf{B}G$.

Par exemple, la catégorie $\mbf{B}\{1\}$ (où $\{1\}$ est le groupe trivial) est
$\mbf{1}$ et la catégorie $\mbf{B}V_4$ (où $V_4$ est le 4-groupe de Klein) est
:

\begin{center}
	\begin{tikzcd}
		G \arrow[loop left, "ab"] \arrow[loop right, "b"] \arrow[loop below, "a"] \arrow[loop above, "\msf{1}"]
	\end{tikzcd}
\end{center}

\begin{definition}[Catégorie opposé]
	Soit $C$ une catégorie.
	La catégorie opposée $C^{\msf{op}}$ possède les mêmes objets et morphismes que $C$.
	Soit $f^{\msf{op}}$ le morphisme dans $C^{\msf{op}}$ correspondant au morphisme dans $C$ :
	\begin{itemize}
		\item[(i)] Le domaine de $f^{\msf{op}}$ est le codomaine de $f$ et le codomaine de $f^{\msf{op}}$ est le domaine de $f$.
		\item[(ii)] Pour chaque objet $X$, le morphisme $\msf{1}_X^{\msf{op}}$ est le morphisme identité dans $C^{\msf{op}}$.
		\item[(iii)] Pour définir une composition, on peut observer qu'une paire de morphismes $f^{\msf{op}}$, $g^{\msf{op}}$ dans $C^{\msf{op}}$ peut se voir appliquer une composition précisément lorsque la paire $f, g$ est composable dans $C$.
	\end{itemize}
\end{definition}

Par exemple, si $G$ peut être vue comme un groupoïde \sidenote{ Un groupoïde
est une catégorie dans laquelle tous les morphismes sont des isomorphismes.

Par exemple, nous pouvons étendre la notion de groupe fondamental au sens
catégorique, pour n'importe quel espace $X$, son groupoïde fondamental
$\Pi_1(X)$ est une catégorie dont les objets sont les points de $X$ et dont les
morphismes de $x$ à $y$ sont les classes d'homotopies $[\gamma]$ des
applications continues $\gamma : [0,1] \rightarrow X$ dont les points
d'extrémités correspondent à $x$ et $y$. La composition correspond à la
concaténation de chemins et est associative sous équivalence homotopique.

\vspace{5pt}

Puisque $\Pi_1(X)$ est un groupoïde, on peut définir le groupe fondamental
$\pi_1(X, x)$ comme le groupe des automorphismes de $x$ dans $\Pi_1(X)$
\[ \pi_1(X, x) := \mathsf{Aut}_{\Pi_1(X)}(x) \]
Nous verrons dans la partie concernant les foncteurs que ces deux constructions
sont des foncteurs de catégories. } à un seul objet, sa catégorie opposée
$G^{\msf{op}}$ peut également considéré de cette manière. Ce groupe est appelé
le \flqq{} groupe opposé \frqq{}, et possède certaines propriétés
intéressantes.

\begin{definition}[Catégorie de morphisme]
	La catégorie de morphisme $C^{\rightarrow}$ d'une catégorie $C$ est définie par $\msf{Ob}(C^\rightarrow) = \msf{Mor}(C)$, i.e., elle a pour objets les morphismes de $C$.
	Un morphisme $g$ de $f : A \rightarrow B$ à $f' : A' \rightarrow B'$ est une paire de morphismes $(g_1, g_2)$ dans $C$ telle que $g_2 \circ f = f' \circ g_1$.

	\begin{center}
		\begin{tikzcd}
			a \arrow[r, "g_1"] \arrow[d, "f", swap] & A' \arrow[d, "f'"] \\
			B \arrow[r, "g_2", swap] & B'
		\end{tikzcd}
	\end{center}

	La composition est définie en plaçant ces carrés commutatifs côtes à côtes.
\end{definition}

\begin{definition}[Sous-catégorie]
	Une sous-catégorie $D$ d'une catégorie $C$ est définie en limitant $C$ à une sous collection d'objets et de morphismes sous réserve que la sous-catégorie $D$ contienne le domaine et le codomaine de tout morphisme dans $D$, le morphisme identité de tout objet dans $D$ ainsi que la composition de tout couples de morphismes dans $D$.
\end{definition}

Par exemple, toute catégorie $C$ possède un sous-groupoïde maximal contenant
tous les objets et morphismes qui sont des isomorphismes. La catégorie
$\msf{Fin}_\msf{iso}$ des ensembles finis avec bijections est la
sous-collection maximum de la catégorie $\msf{Fin}$ des ensembles finis avec
toutes les fonctions.

\section{Types de morphismes}

\marginnote{
	\begin{prop}
		Tout isomorphisme est un bismorphisme.
		Plus généralement, toute rétraction est un monomorphisme et toute section est un épimorphisme.
		Néanmoins, tous les bimorphismes ne sont pas forcément des isomorphismes.
	\end{prop}

	Les deux premières parties de la proposition sont triviales. Un exemple de
	bimorphisme qui n'est pas un isomorphisme est l'inclusion $\mathbb{Z}
		\hookrightarrow \mathbb{Q}$ dans la catégorie des anneaux $\msf{Rings}$. C'est
	bien un bimorphisme mais pas un isomorphisme. }

\begin{definition}[Monomorphismes et Épimorphismes]
	La notion de monomorphisme généralise la notion d'application injective dans $\msf{Set}$ pour n'importe quelle catégorie et est duale à la notion d'épimorphisme qui généralise la notion de surjection.

	\begin{itemize}
		\item[(i)] Un morphisme $f : X \rightarrow Y$ dans une catégorie $C$ est un monomorphisme si pour tout objet $Z$ et toute paire de morphismes parallèles $g_1, g_2 : Z \rightarrow X$ alors \[ (f \circ g_1 = f \circ g_2) \implies (g_1 = g_2) \]
			ou autrement dit que \[ \msf{Hom}(Z, X) \overset{f}{\hookrightarrow} \msf{Hom}(Z, Y) \]

		\item[(ii)] Un morphisme $f : X \rightarrow Y$ dans une catégorie $C$ est un épimorphisme si pour tout objet $Z$ et toute paire de morphismes parallèles $g_1, g_2 : Z \rightarrow X$ alors \[ (g_1 \circ f = g_2 \circ f) \implies (g_1 = g_2) \]
			ou autrement dit que \[ \msf{Hom}(Y, Z) \overset{f}{\hookrightarrow} \msf{Hom}(X, Z) \]
	\end{itemize}
\end{definition}

Un morphisme satisfaisant ces deux définitions est un bimorphisme.

\marginnote{
	En quelques mots, un foncteur est un morphisme entre catégories.
	En particulier, toute catégorie possède un foncteur identité $1_C : C \rightarrow C$.
	Ainsi, nous pouvons définir $\msf{Cat}$ telle que la catégorie de toutes les catégorie petites avec des foncteurs entre elles.

	\hrulefill

	Un exemple de foncteur que nous avons déjà rencontré est avec pour image le
	groupe fondamental d'un espace
	\[ \pi_1 : \msf{Top} \rightarrow \msf{Group} \]

	Le foncteur $F : \msf{Set} \rightarrow \msf{Group}$ ayant pour image le groupe
	libre d'un ensemble est un autre exemple. C'est le groupe dont les éléments
	sont des \flqq{} mots \frqq{} finis dont les lettres sont des éléments $x \in
		X$ ou leurs inverses formels $x^{-1}$, modulo une relation d'équivalence qui
	égalise les mots $xx^{-1}$ avec le mot nul. La multiplication est définie par
	concaténation. }
\begin{definition}[Rétractions et Sections]
	Un morphisme est une rétraction s'il possède un inverse à gauche et une section s'il possède un inverse à droite.
\end{definition}

Notons que dans $\msf{Set}$, les monomorphismes sont exactement des rétractions
(i.e. des fonctions injectives) et les épimorphismes exactement des section
(i.e. des fonctions surjectives).

\section{Foncteurs}

\begin{definition}[Foncteur covariant]
	Un foncteur covariant (ou simplement un foncteur) $F : C \rightarrow D$ entre deux catégories $C$ et $D$ est une application $\msf{Ob}(C)$ \rightarrow $\msf{Ob}(D)$ et $\msf{Mor}(C) \rightarrow \msf{Mor}(D)$ tels que :

	\begin{align*}
		F(f : A \rightarrow B) & = F : F(A) \rightarrow F(B) \\
		F(1_A)                 & = 1_{F(A)}                  \\
		F(g \circ f)           & = F(g) \circ F(f)
	\end{align*}
\end{definition}

\begin{definition}[Catégories isomorphes]
	Deux catégories $C$ et $D$ sont isomorphes , s'il existe des foncteurs $F : C \rightarrow D$ et $G : D \rightarrow C$ qui sont inverses ($F \circ G = 1_D$ et $G \circ F = 1_C$).
	En particulier, un foncteur est un isomorphisme si et seulement si il est bijective sur la classe des des objets et la classe des morphismes.
\end{definition}

\marginnote{
	\begin{prop}
		Les foncteurs préservent la commutativité dans les diagrammes.
	\end{prop}
	Un diagramme dans $C$ est donné par le foncteur $F : J \rightarrow C$, dont le domaine est une petite catégorie.
	Soit $G : C \rightarrow D$, la composante de $GF : J \rightarrow D$ définit l'image du diagramme dans $D$.
}
\begin{itemize}
	\item[$\bullet$] $F$ est injectif sur les objets d'une catégorie si l'application $F_O : \msf{Ob}(C) \rightarrow \msf{Ob}(D)$ est injective.
	\item[$\bullet$] $F$ est surjectif sur les objets d'une catégorie si l'application $F_O : \msf{Ob}(C) \rightarrow \msf{Ob}(D)$ est surjectif.

		\vspace{5pt}

	\item[$\bullet$] $F$ est injectif sur les morphismes d'une catégorie si l'application $F_O : \msf{Ob}(C) \rightarrow \msf{Ob}(D)$ est injective.
	\item[$\bullet$] $F$ est surjectif sur les morphismes d'une catégorie si l'application $F_O : \msf{Ob}(C) \rightarrow \msf{Ob}(D)$ est surjectif.
\end{itemize}

Un diagramme dans une catégorie $C$ est un foncteur $F : J \rightarrow C$ dont
le domaine, la catégorie indexée \sidenote{En pratique, on peut considérer une
	catégorie indexée comme un graphe orienté définissant la \flqq{} forme \frqq{}
	du diagramme.

	Par exemple, pour définir un foncteur avec $\msf{2} \times \msf{2}$ en domaine,
	il suffit de spécifier les images des quatres objets ainsi que leurs morphismes
	associés
	\begin{center}
		\begin{tikzcd}[ampersand replacement=\&]
			a \arrow[r, "f"] \arrow[d, "g", swap] \& b \arrow[d, "h"] \\
			c \arrow[r, "k", swap] \& d
		\end{tikzcd}
	\end{center}

	sujets à la relation $hf = kg$ } , est une catégorie petite. Un diagramme est
typiquement représenté en dessinant les objets et les morphismes dans son image
(avec la catégorie domaine implicite).

\vspace{5pt}
Considérons $\msf{2} \times \msf{2}$, la catégorie avec quatre objets et les morphismes (non-identités) suivants :
\begin{center}
	\begin{tikzcd}
		● \arrow[r] \arrow[d] & ● \arrow[d] \\
		● \arrow[r] & ●
	\end{tikzcd}
\end{center}

Dans $\msf{2} \times \msf{2}$, le morphisme diagonal est à la fois la
composition des morphismes haut et droit et des morphismes gauche et bas. Un
tel diagramme est dit un carré commutatif et de manière générale un diagramme
est dit commutatif lorsque tous ses chemins avec les mêmes points de départ et
d'arrivés donnent un même résultat.

\hrulefill

\begin{prop}[Théorème du point fixe]
	Toute endomorphisme continu d'un 2-disque possède un point fixe.
\end{prop}

% TODO: diagram
\marginnote{
	\mplibforcehmode
	\mplibtextextlabel{enable}
	\begin{center}
		\begin{mplibcode}
			path circle; circle = fullcircle scaled 2cm;
			draw circle;
			dotlabel.top("$f(x)$", (-0.3cm, -0.4cm));
			path arr; arr = (-0.3cm, -0.4cm) -- (1.6cm, 0.2cm);
			drawarrow arr;
			dotlabel.top("$x$", point 1/3 of arr);
			dotlabel.lrt("$r(x)$", arr intersectionpoint circle);
		\end{mplibcode}
	\end{center}
}

Cela peut être déduit plus ou moins immédiatement à partir de la fonctorialité
du groupe fondamental \[ \pi_1 : \msf{Top} \rightarrow \msf{Group} \]
Soit $f : D^2 \rightarrow D^2$ tel que $\forall x \in D^2,\ f(x) \neq x$, on
peut définir une fonction continue $r : D^2 \rightarrow S^1$ qui transporte un
point $x \in D^2$ à l'intersection de la tangente de $f(x)$ à $x$ avec la
frontière de $S^1$. Ainsi, c'est une rétraction de l'inclusion $i : S^1
	\hookrightarrow D^2$, ce qui revient à dire que \[ S^1 \xrightarrow{i} D^2 \xrightarrow{r} S^1 \] est une fonction identité. \vspace{5pt}

\marginnote{
	\begin{prop}
		Les foncteur préservent les isomorphismes.
	\end{prop}

	Considérons un foncteur $F : C \rightarrow D$ et un isomorphisme $f : x
		\rightarrow y$ avec un inverse $g : y \rightarrow x$. En appliquant
	l'axiomatique des foncteurs, nous obtenons :
	\[ F(g)F(f) = F(gf) = F(1_x) = 1_{Fx} \]

	Ainsi, $Fg : Fy \rightarrow Fx$ est un inverse gauche de $Ff : Fx \rightarrow
		Fy$. Par dualité, $Fg$ est un inverse droit. }

En prenant n'importe quel point de base sur la frontière de $S^1$ et en
appliquant $\pi_1$, on obtient la paire d'homomorphisme de groupes : \[ \pi_1(S^1) \xrightarrow{\pi_1(i)} \pi_1(D^2) \xrightarrow{\pi_1(r)} \pi_1(S^1) \]

Par les axiomes de fonctorialité, nous devons avoir \[ \pi_1(r) \cdot \pi_1(i) = \pi_1(ri) = \pi_1(1_{s^1}) = 1_{\pi_1(S^1)} \]

Cependant, on sait que $\pi_1(S^1) = \mbbf{Z}$ tandis que $\pi_1(D^2) =
	\msf{0}$, le groupe trivial. La composition d'endomorphismes $\pi_1(r) \cdot
	\pi_1(i)$ de $\mbbf{Z}$ doit être $\msf{0}$, puisque cela se factorise à
travers le groupe trivial.

Ainsi, cela ne peut valoir le morphisme identité qui associe le générateur $1
	\in \mbbf{Z}$ à lui-même ($0 \neq 1$). Cette contradiction démontre que la
rétraction $r$ ne peut exister et que $f$ doit avoir un point fixe.

\begin{definition}[Foncteur contravariant]
	Un foncteur contravariant $F$ de $C$ à $D$ est un foncteur $F : C^{\msf{op}} \rightarrow D$.
	Un morphisme dans le domaine du foncteur $F : C^{\msf{op}} \rightarrow D$ sera toujours représenté par une flèche $f : c' \rightarrow c$ dans $C$.

	\begin{center}
		\begin{tikzcd}
			C^{\msf{op}} \arrow[r, "F"] & D
		\end{tikzcd}
	\end{center}

	\marginnote{
		\vspace{-2cm}
		\begin{definition}
			Soient $C$ et $D$ des catégorie, alors il existe une catégorie produit $C \times D$ dont :
			\begin{itemize}
				\item[(i)] les objets sont des paires ordonnées $(c, d)$ où $c$ est un objet de $C$ et $d$ un objet de $D$
				\item[(ii)] les morphismes sont des paires ordonnées $(f, g) : (c, d) \rightarrow (c', d')$ avec $f : c \rightarrow c' \in C$ et $g : d \rightarrow d' \in D$
			\end{itemize}
		\end{definition}
	}

\end{definition}

\begin{itemize}
	\item[(i)] Le foncteur contravariant $P : \msf{Set}^{\msf{op}} \rightarrow \msf{Set}$ associe un ensemble à son ensemble de partitions $P(A)$ et une fonction $f : A \rightarrow B$ à son image inverse $f^{-1} : P(B) \rightarrow P(A)$ qui associe $B' \subset B$ à $f^{-1}(B') \subset A$.
	\item[(ii)] Soit $C$, une catégorie petite, un foncteur $C^{\msf{op}} \rightarrow \msf{Set}$ est un \flqq{} préfaisceau\sidenote{Nous y reviendrons dans la section consacrée aux faisceaux.} \frqq{} sur $C$.
		Un exemple typique est le foncteur $\mcf{O}(X)^\msf{op} \rightarrow \msf{Set}$ dont le domaine est l'ensemble partiellement ordonné $\mcf{O}(X)$ des sous-ensembles ouverts d'un espace topologique $X$ et dont la valeur à $U \subset X$ est l'ensemble des fonction réels continus sur $U$.
		L'opération sur les morphismes est pas restriction.
		Ce préfaisceau est dit un \flqq{} faisceau \frqq{}.
\end{itemize}

\section{Transformations naturelles}

Considérons la catégorie $\msf{Vect}_\Bbbk^{\msf{fd}}$ des $\Bbbk$-espaces
vectoriels de dimension finie. Tout objet $\mcf{V} \in
	\msf{Vect}_\Bbbk^{\msf{fd}}$ est isomorphe à son dual linéaire $\mcf{V}^* =
	\msf{Hom}(\mcf{V}, \Bbbk)$ (la dimension de $\mcf{V}^*$ est égale à celle de
$\mcf{V}$). Cela peut être prouvé en construisant explicitement une base duale:
on choisit une base $e_1, \dots, e_n$ pour $\mcf{V}$ et on définit ensuite
$e_1^*, \dots, e_n^* \in \mcf{V}^*$ par
\[ e_i^* (e_j) = \begin{cases} 1 & i = j \\ 0 & i \neq j \end{cases} \]

La collection d'éléments $e_1^*, \dots, e_n^*$ permet de définir une base sur
$\mcf{V}$ et l'application $e_i \mapsto e_i^*$ définit par linéarité un
isomorphisme $\mcf{V} \simeq \mcf{V}^*$.

\vspace{5pt}

Considérons maintenant la construction analogue du double dual $V^{**} =
	\msf{Hom}(\msf{Hom}(\mcf{V}, \Bbbk), \Bbbk)$ de $\mcf{V}$. Si $\mcf{V}$ est de
dimension fini, alors l'isomorphisme $\mcf{V} \simeq \mcf{V}^*$ est transporté
par le foncteur d'espace vectoriel dual \[ (-)^* : \msf{Vec}_\Bbbk^{\msf{op}} \rightarrow \msf{Vect}_\Bbbk \] vers un isomorphisme $\mcf{V}^* \simeq \mcf{V}^{**}$. La composition $V \simeq
	\mcf{V}^{**}$ permet d'associer la base $e_1, \dots, e_n$ à la base 2-duale
$e_1^{**}, \dots, e_n^{**}$. \vspace{5pt}

Néanmoins, cet isomorphisme a une description plus simple. $\forall v : \in
	\mcf{V}$, la « fonction d'évaluation »
\[ f \mapsto f(v) : \mcf{V}^* \xrightarrow{\msf{ev}_v} \Bbbk \] définit une fonctionnelle linéaire sur $\mcf{V}^*$. Il s'avère que
l'affectation $v \mapsto \msf{ev}_v$ définit un isomorphisme linéaire $\mcf{V}
	\simeq \mcf{V}$, cette fois ne nécessitant pas un choix d'une base\sidenote{En
fait, $e_i^{**}(e_j^*) = e_j^*(e_i) = \mcf{ev}_{e_i}(e_j^*)$ et donc les deux
isomorphismes $\mcf{V} \simeq \mcf{V}^{**}$ – c'est seulement notre description
qui s'est améliorée. }.

\marginnote{
	Un isomorphisme naturel est une transformation naturelle $\alpha : F \Rightarrow G$ dans lequel tout composant $\alpha_c$ est un isomorphisme.
	On peut alors écrire $\alpha : F \simeq G$.
}
\begin{definition}[Transformation naturelle]
	Soit $C$ et $D$ des catégories et $F, G, : C \rightrightarrows D$ des foncteurs, une transformation naturelle $\alpha : F \Rightarrow G$ est une famille de morphismes $\alpha_c : Fc \rightarrow Gc$ dans $D$ pour chaque objet $c:C$, telle que, pour tout morphisme $f : c \rightarrow c'$ dans $C$, le carré suivant des morphismes dans $D$

	\begin{center}
		\begin{tikzcd}
			Fc \arrow[r, "\alpha_c"] \arrow[d, "Ff", swap] & Gc \arrow[d, "Gf"] \\
			Fc' \arrow[r, "\alpha_{c'}"] & Gc'
		\end{tikzcd}
	\end{center}

	commute, c-à-d, si $D$ admet un morphisme $Fc \rightarrow Gc'$ \sidenote{ Cette
		condition de naturalité ne peut être énoncé avec moins de précision : elle se
		réfère à chaque objet et à chaque morphisme dans la catégorie de domaine et est
		décrite à l'aide des images dans la catégorie de codomaine. \vspace{5pt}

		Les données nécessaires pour définir une transformation naturelle $\alpha$ sont
		souvent représenté dans un digramme :

		\begin{center}
			\begin{tikzcd}[ampersand replacement=\&]
				C \arrow[r, "F", bend left=40, ""{name=U, below}]{}
				\arrow[r, "G", bend right=35, swap, ""{name=D, above}]{} \& D \arrow[Rightarrow, from=U, to=D]
			\end{tikzcd}
		\end{center}
	}.
\end{definition}

En pratique, il est généralement plus élégant de définir une transformation
naturelle en disant que la collection $X$ des morphismes définit les composants
d'une transformation naturelle, laissant implicitement les choix du domaine et
codomaine des foncteurs, et des catégories source et cible. \vspace{5pt}

Ici $X$ devrait être une collection de morphismes dans une catégorie (cible)
clairement identifiable, dont les domaines et codomaines sont définis en
utilisant une « variable » commune (un objet de la catégorie source). Si cette
variable est $c$, on pourrait affirmer que les morphismes $X$ sont naturels
dans $c$. \vspace{5pt}

Nonobstant, la totalité des données des catégories sources et cibles, de la
paire de foncteurs parallèles, et les composantes doivent doivent toujours être
considérées comme faisant partie de la transformation naturelle.

\begin{itemize}
	\item[(i)] Pour des espaces vectoriels de n'importe quelle dimension, l'application $\msf{ev} : \mcf{V} \rightarrow \mcf{V}^{**}$ qui transporte un élément $v \in \mcf{V}$ à la fonction linéaire $\msf{ev}_v : \mcf{V}^* \rightarrow \Bbbk$ définit les composantes d'une transformation naturelle de l'endofoncteur sur $\msf{Vect}_\Bbbk$ au double foncteur dual.
		Pour vérifier que le carré naturel
		\begin{center}
			\begin{tikzcd}[ampersand replacement=\&]
				\mcf{V} \arrow[d, "\phi"] \arrow[r, "\msf{ev}"] \& \mcf{V}^{**} \arrow[d, "\phi^{**}"] \\
				\mcf{W} \arrow[r, "\msf{ev}"] \& \mcf{W}^{**}
			\end{tikzcd}
		\end{center}
		commute pour n'importe quelle application linéaire $\phi : \mcf{V} \rightarrow \mcf{W}$, il suffit de considérer les images d'un vecteur générique $v \in \mcf{V}$.
		Par définition, $\msf{ev}_{\phi v} : \mcf{W}^* \rightarrow \Bbbk$ transporte une fonction $f : \mcf{W} \rightarrow \Bbbk$ à $f(\phi v)$.
		\hfill
		\textbf{TODO : finir la démo}

	\item[(ii)] En contraste, le foncteur identité et le seul foncteur dual sur les espaces vectoriels de dimension finie ne sont pas naturellement isomorphes.
		En effet, le foncteur identité est covariant tandis que le foncteur dual est contravariant.
		De plus, l'isomorphisme $\mcf{V} \simeq \mcf{V}^*$ peut être défini lorsque $\mcf{V}$ est fini nécessite un choix d'une base, qui n'est essentiellement préservé par aucunes application linéaire (et plus généralement par aucun endomorphisme linéaire autre que l'identité).
	\item[(iii)] Il existe une transformation naturelle $\eta : 1_\msf{Set} \Rightarrow P$ de la fonction identité
\end{itemize}

\section{Limites et Colimites}

À partir d'un espace euclidien $\mbbf{R}$ avec sa métrique habituelle, on peut construire une grande variété de nouveaux espaces.
Par exemple, en prenant les produit de $\mbbf{R}$ avec lui-même, on obtient les espaces euclidiens $\mbbf{R}^n$ en dimensions finies qui possèdent eux-mêmes des sous-espaces intéressants, définis comme étant les solutions de certaines fonctions polynomiales continues $\mbbf{R}^n \rightarrow \mbbf{R}$, y compris la sphère $\mbf{S}^{n-1}$ qui délimite le disque fermé $\mbf{D}^n$.

\vspace{5pt}

A partir de sphères et de disques, on peut construire des tores, le ruban de
Möebius, la bouteille de Klein, et plus généralement tout complexe cellulaire.
Dans chacun des cas, l'objet construit est un ensemble particulier doté d'une
topologie spécifique et toutes ces topologiques peuvent être définies de
manière uniforme, via une propriété universelle qui caractérise l'espace
nouvellement créé, soit comme une limite ou colimite d'un diagramme particulier
dans la catégorie des espaces topologiques.

Les limites et les colimites, peuvent être définies dans n'importe quelle
catégorie.

\marginnote{
	Explicitement :
	\begin{itemize}
		\item[$\bullet$]
			La donnée d'un cône sur un diagramme $F : \msf{J} \to \msf{C}$ avec pour sommet $c$ est une collection de morphismes $\lambda_j : c \to F_j$ indexés par les objets $j \in \msf{J}$.
		\item[$\bullet$]
			Une famille de morphismes $(\lambda_j : c \to F_j)_{j \in \msf{J}}$ définit un cône sur $F$ si et seulement si, pour chaque morphisme $f : j \to k$  dans $\msf{J}$, le triangle suivant commute dans $\msf{C}$.
	\end{itemize}

	\begin{center}
		\begin{tikzcd}[ampersand replacement=\&]
			\& c \arrow[ld, "\lambda_j", swap] \arrow[rd, "\lambda_k"] \& \\
			F_j \arrow[rr, "F_f"] \& \& F_k
		\end{tikzcd}
	\end{center}
}
\begin{definition}[Cône]
	Soit $\msf{J}$ une catégorie « indexée » (dans le sens où elle ne possède aucune propriété particulière à la manière d'un ensemble indexé) et $\msf{C}$ une catégorie arbitraire.
	Un diagramme de forme $\msf{J}$ dans $\msf{C}$ est un foncteur $F : \msf{J} \rightarrow \msf{C}$.
	Un cône sur un diagramme $F : \msf{J} \to \msf{C}$ avec pour sommet $c \in \msf{C}$ est une transformation naturelle $\lambda : c \implies F$ où le domaine est foncteur constant à $c$.
	Les composants $(\lambda_j : c \to F_j)_{j \in \msf{J}}$ des transformations naturelles sont les « pattes » du cône.

	\begin{definition}{Limite}
		Pour tout diagramme $F : \msf{J} \to \msf{C}$, une limite est un objet terminal dans la catégorie des cônes sur $F$, c.-à-d., $\int Cone(–, F)$.
		Un objet dans $\int Cone(–, F)$ est un cône sur $F$ pour n'importe quel sommet.
		En particulier, la donnée d'un objet terminal dans $\int Cone(–, F)$ est un objet limite accompagné d'un cône limite.

		Un morphisme d'un cône $\lambda : c \implies F$ vers un cône $\mu : d \implies
			F$ est un morphisme $f : c \to d$ dans $\msf{C}$ tel que pour tout $j \in
			\msf{J}$, $\mu_j \cdot f = \lambda_j$. En d'autres termes, un morphisme de cône
		est une application entre les sommets telle que chaque « patte » du cône
		domaine est translatée à travers la « patte » correspondante dans le cône
		codomaine.
	\end{definition}
	\begin{center}
		\begin{tikzcd}
			& & & c \arrow[llldd, "\dots", swap] \arrow[lldd, "\lambda_{-2}" description] \arrow[ldd, "\lambda_{-1}" description] \arrow[rdd, "\lambda_1" description] \arrow[rrdd, "\lambda_2" description]\arrow[rrrdd, "\dots"] \arrow[d, "f"] \arrow[dd, bend left, "\lambda_0" description] & & & \\
			& & & d \arrow[rrrd, "\dots"] \arrow[rrd, "\mu_2" description] \arrow[rd, "\mu_1" description] \arrow[llld, "\dots"] \arrow[lld, "\mu_{-2}" description] \arrow[ld, "\mu_{-1}" description] \arrow[d, "\mu_0" description] & & & \\
			\dots \arrow[r] & F(-2) \arrow[r] & F(-1) \arrow[r] & F(O) \arrow[r] & F(1) \arrow[r] & F(2) \arrow[r] & \dots
		\end{tikzcd}
	\end{center}

	%Un cône sur un diagramme $F : \msf{J} \rightarrow \msf{C}$ est un objet $N \in \msf{C}$ avec une famille de morphismes $\psi_X : N \to F(X)$ indexés par les objets $X$ de $J$ tel que pour tout morphisme $f : X \to Y$ dans $J$, on a $F(f) \circ \psi_X = \psi_Y$

	\vspace{5pt}

\end{definition}

\end{document}
